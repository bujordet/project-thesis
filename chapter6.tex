%===================================== CHAP 5 =================================

\chapter{Conclusion} \label{cha:conclusion}
The previous chapter analyzed and discussed the results of the research conducted in this project, as well as discuss prior research and experience identified in the literature study. This chapter aims to conceptualize the knowledge and results in a conclusion, answering the questions asked in section \ref{sec:rq1} and \ref{sec:rq2}. Furthermore, giving a direction for further explorations in section \ref{sec:further_research}.

\noindent The main research question of this projects was: 
\begin{quote}
    \textit{What is the fundamental reason for the striking difference in labor productivity between the ICT-industry and construction industry?} 
\end{quote}

\noindent Which was broken into two sub-questions:

{\noindent \bf RQ1:} Why does the difference in LP between the ICT-industry and CI appear? \\
{\bf RQ2:} How does the difference appear in the LSB-project, which utilize both agile and digital tools?

Section \ref{sec:rq1} and \ref{sec:rq2} outline the answer of RQ1 and RQ2 accordingly.

\section{A Industry Arranged for the Past} \label{sec:rq1}
The literature study identified an industry utilizing processes far behind its counterparts with an ever-increasing complexity hard to grasp with traditional processes and without aiding software. Most of the problems promote an agile methodology to be used in parts of the process. 

Based on a case study researching a project utilizing lean and digital tools. The project has utilized a qualitative analysis of the case study, the research has identified three themes discussed in section \ref{sec:dis_interaction}, \ref{sec:dis_responsibility}, and \ref{sec:dis_understanding} of the \nameref{cha:discussion} chapter. These are the problems causing the difference in LP, discovered in this project.

{\bf Problems in the interaction between the personnel:} To make the project run as smooth as possible, the people working have to cooperate. This is difficult without proper communication and interaction between different parties. If not cooperating properly, productivity decreases drastically.

{\bf Problems in decision making:} Deciding on more or less critical questions is the root of much time spent. Often the problem is rooted in role complications. All the time spent on not deciding is fundamental in the LP gap.

{\bf Problems in ignorance:} The industry has become more complex, making the industry taking action. This causes a rapid change in methods and tools. Following these changes are, therefore, difficult, causing misinterpretation and misuse of what could be useful measures.

\section{Problems hampering the new methods} \label{sec:rq2}
The discovery of the three themes in the previous section is the formation of two key underlying reasons for the LP issue to be present, in a project like the LSB-project, which utilizing both agile and digital tools. 

\begin{enumerate}
    \item The actors' traditional perception interaction, transparency, and contracts hamper with the project's ability to utilize the strength of lean. The way LPs shelter the modellers from the blackboard-meetings are examples of how interaction and trancparency occur, because one are more concerned protecting their contracts.
    \item The actors limited understanding of the lean methods because of the lack of previous experience. The older PLs utilizing old tools such as Gantt charts are examples of this. 
\end{enumerate}

These two reseans are important for one another. If the understanding of lean had been good enough one could have understand the limitaions the traditional perception causes.

\section{Further Work} \label{sec:further_research}


\cleardoublepage