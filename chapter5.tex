%===================================== CHAP 5 =================================

\chapter{Discussion} \label{cha:discussion}
This chapter discuss the questions on the research question, and discuss the results. Using knowledge from both the litterature review and the case study. This discussion will give the foundation for the conclusion and further work discussed in chapter \ref{cha:conclusion}. 

This project thesis tries to identify the fundamental reason for the striking difference in LP between the ICT-industry and CI, in a project utilizing lean methodology and software supporting it. The results indicate that the use of tools as lean and groupware is not as helpful as hoped, due to reduced utilization. The indication for poor LP identified in the LSB-project has a basis in the interaction between the workers, where a lack of understanding and unsure responsibility are a significant impact on the projects productivity. 

This result suggests the reason for the striking difference in LP. This chapter will discuss this result around the three key themes identified in the analysis of the interview data. Furthermore, look at different approaches defeating these issues, as well as look at the limitations of this thesis.

\section{Interaction} \label{sec:dis_interaction}
The interaction in a complex organization, with lots of contractors, causes a challenge for the interaction. A monumental issue is the contracts themself. The entrepreneurs are often less eager to cooperate if this could weaken their contract. Furthermore, the fear of transparency is a true barrier for good cooperation and interaction to happen. This cost of this issues is in line with Zaghoul \cite{zaghloul2003construction}, indicating the increased total cost for the project. Moreover, the contracts are in practice a concern about risk allocation. With the unsure responsibility, investigated further in section \ref{sec:dis_responsibility}, proclaiming an even more significant increase in total cost. 

The vision and strategies are an essential aspect of the management of the organization. The results contradict the claims of Buvik and Rolfsen \cite{rolfsen} that the development of a common philosophy: namely the vision, will aid the trust among team members. The problem in the LSB-project is that the top-down approach heavily influences the management. This gives workers who have little perception of the common philosophy. Furthermore, early and clear role expectations and early development of trust are problematic in a project where there is a high degree of turnover. 

The top-down approach makes for a problematic implementation of a common philosophy, but also the implementation of lean. One of the principles of the agile manifesto \cite{agile_manifesto} states:
\begin{quote}
    \textit{Build projects around motivated individuals. Give them the environment and support they need, and trust them to get the job done.}
\end{quote}

This principle is contrary to the top-down approach influencing the project's structure. Also, how the lean is design is supporting the old methods used, and often contrary to the agile manifesto. The use of Gantt-chart is a prime example of this issue. 

When they are applying lean in a top-down manner, it leads to the workers working in the same way as done before. This observation contradicts the claims from Ingvaldsen and Rolsen that the introduction of lean can hamper the Norwegian working model. This contradiction supports the difference between ICT-industry and CI. In the ICT-industry, the workers claim that they are working lean, while in the CI, lean is more a tool for project management.

\section{Lack of understanding} \label{sec:dis_understanding}
Construction engineering is complicated and even more complicating when needing to synchronize with all other disciplines needed to get the building finished. The amount of knowledge needed in this business is immense. It is also wrong to state that the business lacks the understanding of computers and lean methodology, on a general basis. Though, in the LSB-project, there have been identified issues concerning lack of understanding, when introducing the tools used. 

When considering the lean methodology, one might argue that it is not that they do not understand it, but that they will not use it. The reason might be that they have not seen the effect, and while knowing the old methods by heart, it is more natural to use those. On the other hand, when using it, they argue they are lean, just because they use the tools applied. Working on the underlying aspects of lean is, therefore, forgotten. Also, the motivation for utilizing lean is vague. When they do not grasp the motivation for utilizing lean, other than that, others have used it, and it was a success on previous projects. It seems like they have not understood the fundamental principles. 

I addition to misinterpretation of lean, the understanding of digital tools is weak. In some cases, the digital tools applied are more a burden for the users than actual help. There are two main explanations for this problem to occure. One is that the tools investigated are for the most used by the managers, directors, and PLs, which tend to be of the older generation. The technical skillset is thus lower than the ones younger. Also, changing from one project to another with all different programs makes it even more difficult. Second, the usability of software applied is often not too good. The focus is more on the functionality of the program, rather than usability. 

Both of these cases fit with the theory of a bottom up approch will be more sufficient securing the use of a certain tool \cite{Robey&Sahay}. While the top-down approach promotes are more broad use of the methodology, and can be sufficient for other projects later on, because the management can take the knowledge, and lessons learnd, and use it later \cite{lean_i_praksis}. Though, it seems like a bottom-up approach will be most effective, the initiative is hard to find in most projects and thus the top-down approch is needed implementing a new way of thinking.

Contradict to other research where one examine how users react to software improving (for the users) allready known practices \cite{o1999home,svanaes2010usability, nygren1992reading}. The use of Cogito is not. Since Cogito is a tool to support the lean construction and lean design methodology, users not familiar with does do not have the prior context and understanding, which makes it even more difficult to use. Thus, using a top-down approch seems legit. The software supports only the project management, and thus not help in transparency and coordination between modelers, architects and workers.

\section{Unsure resposibility} \label{sec:dis_responsibility}
The LSB-project makes use of a customized contract, named \textit{Totalentreprise med forutgående samspill}. The implications of a new contract have shown to be a more complicated issue than the intention. The intention of the contract was for the contract to support the interaction and prevent errors in the design. Also, considering the project consist of about 30 different contracting firms. The result is a problem in harmonization between the actors, thus leads to difficulty utilization of lean. This argument correlates with the previous research \cite{miller2002harmonization}.

The project seems to have clear role expectations, as Rolfsen promotes\cite{rolfsen}. The results show that even though the project has clear role expectations, it seems like the managers and PLs have problems in decision making. This might suggest that the contractors are up to secure their contracts and increase their revenue. However, based on prior research, one might argue that a more plausible explanation has to do with risk allocation \cite{zaghloul2003construction}. 

The implications of the insecurity lead to delays and reduced utilization of human resources, which again leads to low LP.

\section{Recommendations}
The goal of this thesis was to identify the fundamental reasons for the striking gap in LP between the CI and the ICT-industry. The case seen in the previous chapter is a pioneer project due to its outspoken strategies and the use of new contracts and tools. In the research, this thesis has both found answers to the project question, as well as identified some recommendations. These recommendations face itself as further studies, but also actions for the LSB-project to make. This section is to discuss these findings.

The CI heavily invests in BIM, so do the LSB-project. The BIM maturity level 3 indicates, as mentioned in section \ref{sec:bim}, a shared BIM-resource. That is, there is a common model, used by everyone and updated so that everyone has the same knowledge about the model at every time. In the LSB-project, they do have a shared model, but this is only updated once a week. Having a shared model using a cloud-based system does not imply efficiency. Utilizing the system is demanding, and the correct tools are needed. Though the low utilization could be the symptoms of keeping up with trends \cite{rolfsen2004tyranny}, and not fully understand the potential of the provided tools. Having a shared model, does not help if they do not utilize the potential.

With today's technology, BIM updates could happen on the spot. As mentioned in the literature review, using a SaaS architecture could fulfill a real-time cloud-based system, providing every stakeholder the required information whenever needed. Also, having all BIM-software, including dRofus, use the shared resource will enhance productivity. Revit does not support this; thus, other measures are needed to be taken. Increasing the frequency of uploads could aid the interaction, though further investigations are needed to identify the implications of increasing such frequency. Furthermore, new updated tools are needed to make full use of what state-of-the-art cloud-architecture has to offer. Further research about BIM in an lean project could give answers to the 

In the LSB-project, there is some software offered by the project managers, which are used commonly by all actors. Furthermore, every division has its specialist software needed for their specific job. All these programs often imply some authentication, which causes a challenge for some actors. Also, when introducing new software, for some, that is just another password and username. Thus, the need for a single-sign-on (SSO) is present. 
An SSO is giving the user access to all protected resources, after authenticated once using one single authentication mechanism, without reauthenticating. The Open Group \cite{open2019} defines SSO as: 
\begin{quote}
    \textit{The mechanism whereby a single action of user authentication and authorization can permit a user to access all computers and systems where that user has access permission, without the need to enter multiple passwords.}
\end{quote}
The need for SSO is especially present for the specialist, who is working on several projects simultaneously. Having a single-sign-on integrated with every tool needing authentication will improve efficiency dramatically. The effect of using SSO is general, thus not answer the question of this research, but this thesis recommends making use of an SSO, especially with the use of a SaaS-BIM-service. 

Cogito is a significant support in interaction in the LSB-project. The tool offers transparency for project managers, and as a boundary object supporting knowledge sharing. Even though some of the PLs and managers do not like the transparency all that much, the tool is shown effective. Though the tool is effective, the potential is greater. Making workers, modelers, and architects, responsible for the task using Cogito, will make them transparent to work needed to be done.
Moreover, leave some of the responsibility of the PLs. Thus, the research proposes infusing Cogito at a higher degree in the LSB-project. Research promotes a slow deployment of a new tool, also starting with one team at the time \cite{Robey&Sahay}. This will give a more robust deployment, and the first users can work as key personnel later in the deployment, aiding others with their knowledge. Further research is needed to explore the effect of the infusion, also to further explore the research question. Hence, the use of task boards is more common in ASD, such as scrum-board \cite{sutherland}. Thus, the use of Lean can feel more like an activity for the workers, modelers, and architects, more than a project management methodology. 

\section{Limitations}
This project thesis has the goal of giving a direction for the upcoming master thesis. Even though the research is not to be published, the conduction of the results should be equally good. 

The research question tries to identify answers to a generalized problem. Using a single case-study has the limitation of not providing the generalized answer, the method, though, will give a more in-depth observation of the phenomenon. This approach may answer some of the problems, but not all. Thus, the object observed could have displayed some of the issues, but certaynely not all.

Furthermore, the data used in the analysis is, arguably, too small. This is caused by poor planning, both from the researcher's side, but also, the contact at Statsbygg. When the planned observation was done, the project was at the end of a significant milestone, making for the small number of meetings observed. Also, the contact has been sick leave, which made for problematic planning of interviews. The collection of interviewees does not contain any workers (modelers, engineers, architects, etc.), which makes for a difficult discussion of the top-down approach due to misbalance in data.

Despite these limitations, the findings are still quite substantial. The issues reported are present, and some modifications are needed in the CI. Also, the research question has been answered. The finding can arguably describe some of the reasons for the difference in LP in the ICT-industry and CI. Still, further research is needed to establish a definite truth about the situation.

\cleardoublepage