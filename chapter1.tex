%===================================== CHAP 1 =================================
\chapter{Introduction}

This chapter introduce the project, and will give overview over topics and deliveries of this project. The \nameref{sec:background} section will give an introduction to why this research is usefull, as well as the motivation. Next, the \nameref{sec:research} section introduce questions to be answered in this question. Last, the \nameref{sec:thesis} section describe the outline of this project.

\section{Background and Motivation} \label{sec:background}
The Construction Industry (CI) has been a significant part of engineering throughout history. Over the past century, the requirements of constructions have become more and more complex \cite{wood2009factors}. The buildings are getting higher, the tunnels are getting longer, and the roads are getting wider. Sure, the size of things is not equal to the complexity of the construction; however, when considering automated systems, multipurpose functionality, and multiple communication platforms – the complexity is increasing. The increased complexity leads to a significant decline in labor productivity (LP), seen over the past two centuries, mentioned in the article written by SSB \cite{productivity}. As well, managing these projects is much more intricate then it used to, because of the increased numbers of actors participating in the project. 

One can argue that the negative progress in LP in the CI has to do with the increasing complexity, and therefore not a number to consider. Even so, better productivity and efficiency are always something management dicier, simply because of improved marginal cost. Therefore, this study is interesting for managers from other industries than only construction and ICT.

The challenges the CI is experiencing, as well as the process used, are highly similar to what the ICT-industry was facing in the late '80s. The ICT-industry, using the waterfall process \cite{royce}, often faced the challenge of meeting the budgets and timelines. This breach had the origin in change of requirements during production, challenges in testing, and resultingly failing to deliver a finished product without bugs. These problems have been frequently present when creating large and highly user-interactive software — making for the introduction of agile software development to manage theses problems. Over the years, most of the process- and method-management in ICT is digitilized. Giving tools in which both the software developers and project managers use to aid project progression.  

One has often turnd to software when wanting to improve productivity, and hereunder knowledge sharing and interaction in a organization. The implementation of software in a large, complex organization is discussed by many and is shown hard to do well. In particular, the description of the top-down contra bottom-up strategy in implementation \cite{Robey&Sahay}, promotes the importance of making slow change supported by the users. Furthermore, the intention of increasing productivity, by deploying new software is argued by Hammer, to be less sufficient \cite{hammer1990reengineering}. Hammer promotes changing the process of work, rather than improving bits by pieces using specific software. Morover, introduction of software supproting collaboration is shown to be difficult, important in this context is that software break with the social taboos, and adoptation is, as mention, difficult \cite{Grudin}. 

Frank Garry, in 1997, first introduced 3-D modeling in CI, when constructing the Peter B. Lewis Building (PLB). 3-D modeling was used both to manage the complexity of the installation, but also led to increased cooperation between different parties within the project. The paper, describing this project \cite{frank_gehry}, is reporting a change in how actors in the construction react to using computer-aided constructions, in 3-D. Today 3-D modeling is used in almost all construction projects and is known as BiM. Even though the PLB-project showed promising results in means of cooperation and interaction, the introduction of 3-D modeling was not a single solution to the problem.

Furthermore, one has introduced Lean in the CI. A book \cite{lean_i_praksis} describing the making of the Bergen Academy of Art and Design-building, where Lean was one of the essential strategies. The case object of the case study in this research are using experience from this book when managing the constructions. 

The motivation for this rearch is, therefore, to examine a construction project utilizing Lean in project management. Furthermore, looking at how a project make use of digital tools, aiding Lean has not been examined before. Taking experience from the ICT-industry, and the use of computer-aided agile development management is also desirable, as well as looking at the problem from a different perspective.

\section{Research and Question} \label{sec:research}
Based on the background and motivation the research of this project tries to identify the phenomenon cousing the poor PL in CI. The main research question is, therefore:
\begin{quote}
    \textit{What is the fundamental reason for the striking difference in labor productivity between the ICT-industry and construction industry?} 
\end{quote}
This is then broken down to two sub-questions, which this project thesis tries to answer, using a case study of the Life Science Building project.

{\noindent \bf RQ1:} Why does the difference in LP between the ICT-industry and CI appear?

{\noindent \bf RQ2:} How does the difference appear in the LSB-project, which utilize both agile and digital tools?

\section{Thesis structure} \label{sec:thesis}

{\noindent \bf Chapter 2: Literature Review} provides an overview of key findings, concepts and development relevant for the research question. Furtheremore, support the discussion as well as the case. 

{\noindent \bf Chapter 3: Empirical Review} gives an introduction of the case, and impede how they make use of digitalization in their Lean Strategy, as well as how this support cooperation and knowledge sharing in the organization.

{\noindent \bf Chapter 3: Method description} describe the mothodology used in the project. The methodology descriptin describe and discuss the approach, data collection as well as method of anaylysis of the genereated data. Also, an evealuation of the method i provided.

{\noindent \bf Chapter 4: Case review} gives an introduction of the case, and the domain, as context for the project. Furhteremore, descibe and discuss the result of the anaylysis of the case data. 

{\noindent \bf Chapter 4: Discussion} takes the data form the case study, and discuss the results with prior research identified in the literature review. The chapter is outlined by three themes discovered in the analysis of the case data. Furthermore, the evaluation of the project limitation can be found in this chapter.

{\noindent \bf Chapter 5: Conclusion and Further work} answer the rearch questions raised in the \nameref{sec:research} section. Furtheremore, proposing furhter work for the master thesis.


\cleardoublepage